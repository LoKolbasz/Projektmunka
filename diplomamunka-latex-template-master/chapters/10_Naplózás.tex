\label{sec:naplozas}
\subsection{Felhasználási területek}
Naplózás szükséges az eszköz által érzékelt adatok gyűjtésére a protokoll felderítési fázisban.
Emellett segíthet a különböző hibakeresési folyamatokban, ha a program jelezni tudja, hogy éppen mit csinál, milyen hibákba ütközött.

\subsection{Naplózó kimenet: USART}
Az Arduino nem rendelkezik képernyővel. Látható kimenetként egyedül beépített LEDekkel rendelkezik.
Lehetséges az eszközre egy LED kijelzőt helyezni, viszont egyszerűbb, és célszerűbb a beépített soros (USB) porton személyi számítógépre (PC) küldeni az adatokat.
A HAL-ban megtalálható USART vezérlővel, és a µFMT formázóval erre a kimenetre küldött adatok a megfelelő, olvasható formában fognak megérkezni.

\subsubsection{µFMT}
Az µfmt, egy (6-40-szer) kisebb, (2-9-szer) gyorsabb és pánik mentes alternatívája az alapértelmezett formázónak.\cite{ufmt_rust}
Ezek a tulajdonságai nagyon kedvezővé teszik, az erőforrásokban korlátolt ATmega2560 processzorral történő használatra.
Azonban a "Log" interfész metódusainak argumentumai nem támogatják ezt a formázót, ezért azok a metaadatok nem formázhatóak ez által.
Ennek az az eredménye, hogy a méret beli előnyök kevésbé jelentősek, hiszen mindenképpen mellékelni kell az alapértelmezett formázót.
A kiküldendő üzenet végső formázását azonban már átveszi ez a modul, ahol kihasználásra kerül az alacsonyabb futási idő, a dolgozat specifikus metaadat (rögzítés ideje), a standard formázóval már formázott metaadat és az üzenet egymáshoz illesztésekor.

\subsection{Metaadat}
Ahhoz, hogy az üzenet kontextusáról részletes információt tartalmazzon egy üzenet, az üzenethez automatikusan hozzáadhatóak különböző meteadatok, mint például, hívási hely, fontossági szint (hiba, info, stb.).
Az ilyen helyzetek kezelésére jól használható a "log" Cargo csomag, amelyben található a Log interfész. 
Ennek az interfésznek az implementációjával használhatóvá válnak (szintén a csomag által nyújtott) naplózó makrók.
Ezek automatikusan közlik a Log implementációval a hívás kontextusát, amit az implementáció szükséges szerint az üzenethez ad.
A \ref{ora}-as szekcióhoz hasonlóan ez a megvalósítás is egy külön csomagként használható, moduláris.

\subsection{Adatszerializálás}
A tűkön beolvasott adatokat a személyi számítógépre küldés ellőtt célszerű szerializálni, tehát egy standard formátumra formázni, annal érdekében, hogy az adatok biztosan kompatibilis, helyesen beolvasható formában legyenek.
Ehhez egy széles körben támogatott formátum a JSON (JavaScript Object Notation).
Ezt a formátumot a legjelentősebb adatfeldolgozáshoz használt programozási nyelvek (mint például a Python) támogatják.

\subsubsection{Serde}
%Serde is a framework for serializing and deserializing Rust data structures efficiently and generically.
"Serde egy keretrendszer Rust adatstruktúrák hatékony és általános szerializálására és deszerializálására"\cite{serde_rust}.
Ez a keretrendszer több, a dolgozat szempontjából szükségtelen, formátumot is támogat, szükségtelen funkciókkal rendelkezik.
Ezek a programrészek azonban fölöslegesen beleforfulnak a végső futtatható állományba.
Ennek következménye az, hogy a linkelés sikertelen lesz, és a futtatható állomány létre sem jön.

\paragraph{Serde-json}
A Serde ezen verziója csak a JSON formátumban lévő adatok szerializációját, deszerializációját támogatja\cite{serde_json_rust}.
Ennek köszönhetően jelentősen csökken a kimeneti bináris mérete, azonban nem eléggé. További hátránya, hogy a standard könyvtár funkcionalitások kikapcsolása után is szükség van egy dinamikus memóriafoglalóra\cite{serde_json_rust}, amely tovább növeli a program méretét, és nem kerül máshol felhasználásra.
Kevesebb helyet foglal, a "serde-json" csomag, mivel az csak a json modult tartalmazza.
Azonban ez továbbra is szükségtelenül tartalmazza a memóriafoglaló modult, amely a program más területein nem kerül felhasználására.
%This version of serde-json is aimed at applications that run on resource constrained devices.
\paragraph{Serde-json-core} "A serde-json ezen verziója azokat az applikációkat célozza, amelyek erőforrás korlátolt eszközökön futnak. [...]
%(De)serialization doesn’t require memory allocations
(De)szerializáció nem igényel memóriafoglalást."\cite{serde_json_core_rust}
Ez a serde legminimálisabb verziója.
Hátránya az, hogy továbbá is tartalmazza a deszerializáláshoz szükséges kódot, ami nem kerül használatra, hiszen a program nem vár JSON formátumban bemenetet.
Ilyen formátumban csak kimenet keletkezik.
Tehát nem adatok JSON-formátumba kerülnek, de nem jönnek létre adatok JSON-formátumból.
Ez további fölöslegesen a programba fordított kódot tartalmaz, és a linkelő továbbá sem jár sikerrel.

\subsubsection{Egyedi szerializáló}
Mivel az említett szerializáló csomagok egyike sem eredményezett futtatható állományt, ezért a legminimálisabb megoldás hasnálata vált célszerűvé.
\paragraph{Írás egyenesen USART kimenetre}
A cél egyszerű adatstruktúrák JSON-formátumban történő kiírása a soros kimenetre USART hasnálatával.
Mivel a HAL-ba beépített USART megvalósítás támogatja a uWrite interfészt, ezért a µfmt formázóval egyenesen a kimenetre lehet írni a karaktereket, ezzel minimalizálva a memóriahasználatot.
Ezt az írót a szerializáló függvény függőségi injekcióval egy módosítható referencia függvényparaméterként veszi át, hogy a kimenet továbbra is használható legyen a hívó által, és írni is lehessen a kimenetre.
\paragraph{JSON formátum referencia}
A kompatibilitás megőrzése érdekében elengedhetetlen betartani a JSON nyílt szabvány előírásait.
A formázás alapjául az ECMA-404 szabvány\cite{json_standard} szolgál.
\begin{figure}[H]
    \center
    \begin{tabular}{@{} p{0.45\textwidth} @{\hspace{0.1\textwidth}} p{0.45\textwidth} @{}}
    \multicolumn{1}{l}{\textbf{Rust Stuktúra}} & \multicolumn{1}{r}{\textbf{JSON Szerializált}} \\
    \hline
    \ttfamily\begin{tabular}[t]{l}
    struct Szemely \{ \\
    \quad nev: String, \\
    \quad eletkor: u8, \\
    \quad aktiv: bool \\
    \} 
    \end{tabular}
    &
    \ttfamily\begin{tabular}[t]{l}
    \{ \\
    \quad "nev": "Kovács János", \\
    \quad "eletkor": 45, \\
    \quad "aktiv": true \\
    \}
    \end{tabular}
    \end{tabular}
    \cite{json_standard}
    \caption{példa: egyszerű struktúra szerializálása}
    \end{figure}
\paragraph{Korlátok}

Ennek a funkciónak a felhasználása korlátozva van a feljegyző modulra, mivel (Serde-hez hasonló kódgeneráció hiányában) minden új adatstruktúra külön implementációt igényel.
A Serde-vel ellentétben nem tartalmaz általános megoldásokat.

\subsection{JSON adatok küldése kérésre}
Az mérőeszköz indításától folyamatosan érkező adatok nagyban megnehezítik azok kezelését.
Ennek elkerülése érdekében alkalmazásra kerül egy kapcsoló, amivel az adatküldés elindítható és szüneteltethető is.
Minden egyes rögzítési periódus adatai saját JSON tömbben helyezkednek el úgy, hogy tömbben nagyobb indexszel rendelkező elem került később rögzítésre.
Ennek köszönhetően a feldolgozás során könnyen elhatárolhatóak az adatok.
A rögzítés során minden egyes mért periódus során mért tevékenység leírása (pl.: mit hajtott végre az objektív) külön dokumentumban, kerül szintén időrendi sorrendben tárolásra.
Ezáltal korreálhatóvá válnak a mért adatok a vizsgált tevékenységgel.

\subsubsection{Gombnyomás érzékelése}
A gomb egy külső nyomtatott áramkör (NYÁK) próbapanel használatával kerül bekötésre.
Erre azért van szükség, mert az Arduino egyetlen beépített gombja a program teljes újraindítására szolgál.
% \begin{figure}[H]
%     \label{fig: gombkotes}
%     ERROR: Missing figure
%     \caption{Gomb bekötése}
% \end{figure}

A gomb lenyomását a program az általános célú be-, kimeneti (GPIO) tűn érzékeli.
Ha a beolvasott érték "logikai alacsony", akkor a gomb le van nyomva, ha az érték "logikai magas", akkor a gomb nincs lenyomva.
Hibalehetőségek csökkentése céljából az adatküldés a gomb lenyomása utáni felengédéskor indul el,
újbóli lenyomáskor az adatküldés leáll, és felengedés után a következő adatküldést elindító lenyomás, felengedés kombinációra vár.
% Maybe add description of spin lock here
\paragraph{Felhúzú ellenállás}
A bemeneti láb, lebegő bemenet esetén nagyon érzékeny.
Emiatt kis, a környezetből származó feszültség, zaj is képes megváltoztatni a beolvasott értéket azt megbízhatatlanná téve.
Egy felhúzó ellenállás használatával bemeneti logikai jel hiányában a lábon az MCU logikai magas értéket olvas, bemenet esetében pedig logikai alacsonyt, ezáltal elkerüli zajból származó bizonytalanságot.
"A bemeneteken chipen belül egy-egy kb. 20..80 $k\Omega$ felhúzó-ellenállás van beépítve,
melyek ki/bekapcsolhatóak."\cite{evans2011arduino}
A bemeneti lábat a HAL megfelelő metódusával egy felhúzó elleneállással működő, digitális bemenetté alakítva kerülnek ezek a beépített  ellenállások használatra, elkerülve extra elektronika hozzáadását.