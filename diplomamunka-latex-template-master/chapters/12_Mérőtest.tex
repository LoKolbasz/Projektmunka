Ez az az eszköz, amihez az arduino csatlakozik, hogy lefülelje az objektív és a kamera közti kommunikációt.
\subsection{Előgyártott makró közgyűrű}
Ennél a megközelítésnél előgyártott makró közgyűrűsorokat használva, azok csatlakozóira ráakaszkodva történik a beolvasás.
2 féle megközelítés lehetséges:
\begin{enumerate}
    \item Egy gyűrűn belül oldalról ráakaszkodás \label{nem_mukszik_csat}
    \item Kettő gyűrű használatával a kettő csatlakozójának külső, hozzáadott összekötésére való ráakaszkodással.\label{mukszik_csat}
\end{enumerate}
Az \ref{nem_mukszik_csat} előnye az, hogy könnyen az objektív továbbra is képes képet vetíteni a fényképezőgép szenzorjára. Lehet vele fókuszálni, de csak közelre.
Hátránya, hogy nagy mértékben a használt közgyűrű eszközön múlik, és azon, hogy mennyire lehet a belső jelhordozókhoz hozzáférni.

A \ref{mukszik_csat} előnye, hogy könnyebben megvalósítható, és kevésbé van a használt közgyűrűnek hatása a megvalósításra.
Hátránya, hogy használatakor nem vetít képet az objektív a kameraszenzorra külső fény beszökése nélkül, ezáltal az kévésbé tud üzemelni.  
\paragraph{Nikon Z}
Használt közgyűrűsor: Meike MK-Z-AF1.

\paragraph{Nikon F}

\subsection{3D nyomtatott csatlakozók}
\subsubsection{Kontakt problémák}
\subsubsection{Áthallás}