\subsection{Adapterváz}
Online közzétett kontaktok nélküli adapterből kerül kialakítása \ref{bajonett_mero}-es szekcióban ismertetettekhez hasonlóan.
\subsection{Rekeszátmérő kar}
Ennek a megvalósítása a dolgozaton kívül esik.
Lehetséges megközelítések:
\begin{itemize}
    \item Direkt meghajtású elektromotor
    \begin{itemize}
        \item Lineáris léptetőmotor
        \item Szervómotor
    \end{itemize}
    \item Filmes gépeknél alkalmazott szolenoidos rendzszer:
    \begin{figure}[h]
        [INSERT FIGURE HERE]
    \end{figure}
\end{itemize}
\subsection{AF-csavar}
Az AF-csavarnak fontos a szilárdsága és az illeszkedése az objektívbe.
Ennek érdekében a sérült Nikon F közgyűrűből kiszerelt mechanizmus használható.
\subsubsection{AF Meghajtás típusa} %Direct vs indirect drive.
Mivel a kisméretű szervó motorok könnyebben elérhetőek és programozhatóak mint a kis méretű léptetőmotorok, ezért a prototípusnál ezek használata van előnyben létesítve.
\subsection{Kontaktok}
\subsubsection{POGO tűk}