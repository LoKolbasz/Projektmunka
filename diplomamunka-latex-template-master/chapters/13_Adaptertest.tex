\subsection{Adapterváz}
Online közzétett kontaktok nélküli adapterből kerül kialakítása \ref{bajonett_mero}-es szekcióban ismertetettekhez hasonlóan.
\subsection{Rekeszátmérő kar}
Ennek a megvalósítása a dolgozat keretein kívül esik.
Lehetséges megközelítések:
\begin{itemize}
    \item Direkt meghajtású elektromotor
    \begin{itemize}
        \item Lineáris léptetőmotor
        \item Léptetőmotor
        \item Szervómotor
    \end{itemize}
    \item Filmes gépeknél alkalmazott szolenoidos rendzszer:
    \begin{figure}[H]
        \centering
        \includegraphics[width=0.5\linewidth]{img/nikon_f60_f60d_n60_n60d_repair.png}
        \caption{Nikon F60-ban használt rekeszátmérő kar mechanizmus\cite{nikon_f60_manual}}
    \end{figure}
\end{itemize}
\subsection{AF-csavar}
Az AF-csavarnak fontos a szilárdsága és az illeszkedése az objektívbe.
Ennek érdekében a sérült Nikon F közgyűrűből kiszerelt mechanizmus használható.
\subsubsection{AF Meghajtás típusa} %Direct vs indirect drive.
Mivel a kisméretű szervómotorok könnyebben elérhetőek és programozhatóak mint a kis méretű léptetőmotorok, ezért a prototípusnál ezek használata van előnyben részesítve.
%\subsection{Kontaktok}
%\subsubsection{POGO tűk}