\subsection{Magyar}
A dolgozatban bemutatott adapter célja az volt, hogy lehetővé tegye a Nikon F-bajonettes, 
különösen az AF és AF-D típusú objektívek teljes funkcionalitású használatát a modern 
Nikon Z-bajonettes fényképezőgépeken. A tervezés során részletesen feltérképezésre kerültek 
a két rendszer közti elektronikus és mechanikus különbségek, valamint azok áthidalásának 
lehetőségei.

Az eredmények alapján megállapítható, hogy megfelelő hardver- és szoftverarchitektúra 
alkalmazásával technológiailag megvalósítható egy olyan intelligens adapter, amely 
visszaadja a régebbi objektívek fontos funkcióit. A projekt teljes körű kivitelezéséhez 
azonban – a dokumentáció hiánya és a kommunikációs protokollok ismeretlensége miatt – 
valószínűleg egy kisebb, több szakterületet lefedő fejlesztőcsapat munkájára lenne szükség. 
Ez az adapter jelentős mértékben hozzájárulhatna a Nikon objektívörökség modern rendszereken 
való továbbéléséhez.

\subsection{English}
The goal of the thesis was to design an adapter that enables full functionality of Nikon 
F-mount lenses—especially AF and AF-D types—on modern Nikon Z-mount mirrorless cameras. 
The design process involved a detailed analysis of the mechanical and electronic differences 
between the two systems and explored ways to bridge those gaps.

The results indicate that, with appropriate hardware and software architecture, it is 
technologically feasible to create a smart adapter that restores key features of older lenses. 
However, due to the lack of official documentation and the proprietary nature of communication 
protocols, the full implementation of such a project would likely require the coordinated effort 
of a small, multidisciplinary development team. Such an adapter could play a significant role in 
preserving and extending the usability of Nikon's legacy lens ecosystem on next-generation platforms.