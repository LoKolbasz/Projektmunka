\subsection{A projekt célja}
Egy olyan adaptert megalkotni, amely képes Nikon F-bajonettes objektíveket Nikon Z bajonettes fényképezőgépvázakra adaptálni, miközben azok megtartják teljes funkcionalitásukat. Bár hasonló adapterek léteznek, egyik sem képes visszaadni a régebbi AF és AF-D megjelölésű objektíveknek az automatikus fókuszáló képességét, amelyet beépített fókuszmotor hiányában a korábbi megoldások nem tartalmaznak. Ettől a hiányosságtól csökken a megtartható fényképek száma a fentebb említett lencsék esetén, ugyanis a fókusztávolságot manuálisan kell állítani, amely lassabb, és jóval nehezebb, mint a mutatás és fotózás (point and shoot). Ennek az új adapternek köszönhetően régebbi csúcsminőségű optikák is teljesen használhatóvá válnak, melyek között vannak olyan típusúak, amelyekből sosem készült új verzió.
\subsubsection{Az F-bajonett rövid története}
Az F-bajonettes objektívek története több mint 50 évre nyúlik vissza. A Nikon Corporation (eredeti nevén Nippon Kogaku K.K.) 1918 óta gyárt objektíveket és 1948 óta kamerákat. A vállalat 1959-ben bevezette a fényképezőgépéhez a Nikon F-hez az új F (immár) bajonettes csatlakozást\cite{Nikon_tori}, amellyel gyorsabban lehetett objektívet cserélni, valamint a csatlakozás is erősebb, precízebb lett. A Nikon tükörreflexes fényképezőgépei a mai napig ezt a bajonettet használják, legyenek azok digitálisak (DSLR) vagy analógok (SLR). A cég az autófókusz bevezetésekor hozott egy olyan döntést, aminek negatív következményét ez a projekt kívánja enyhíteni. Amíg a cég legfőbb riválisa a Canon ekkor új bajonettet vezetett be és a fókuszmotort az objektívbe helyezte el, ezzel szemben a Nikon megtartotta a csatlakozást, és csak módosította úgy, hogy egy csavarral a kameratestbe épített motor tudja állítani az optika fókuszát. Így bár az automatikus fókuszállítási sebesség (főleg a későbbiekben) csökkent, a korábbi autófókusz nélküli objektíveket is lehetett használni autófókuszos fényképezőgépvázakon és fordítva. Később azonban a Nikon vállalat is felismerte, az objektívekbe szerelt fókuszmotorok előnyeit, és a későbbi modellek már ilyen technológiákat használtak.
\subsubsection{A hivatalos Nikon FTZ adapter hiányosságai}
Amikor 2018-ban az új tükör nélküli fényképezőgépeket bemutatták\cite{Nikon_Z_Release}, akkor azok az új, Z bajonettjéhez készült hivatalos F bajonettes objektíveknek Z bajonettes vázra adaptáló adapterbe nem lett beépítve fókuszmotor, amelyekkel az idősebb AF és AF-D megjelölésű optikákat lehetne automatikusan fókuszálni, ami az alábbi táblázaton látható.

Bár könnyen meglehet, hogy ez egy jó pénzügyi döntés volt, hiszen ekkorra már a modern optikák közül egy sem támaszkodott a vázmotorra, de több professzionális fényképész is megkérdőjelezte a döntést. Ez többek között annak is köszönhető, hogy több népszerű és jó minőségű eszköz (pl.: Nikon Af 50mm 1.8(D), és a Nikon DC, (amely különleges, más fajta objektívek által nem nyújtott irányítást adott a kép fókuszon kívüli része felett) funkcionalitásából veszített az új platformon. A csalódottak között volt Ken Rockwell fotoblogger is, aki így fogalmazott: „A szomorú valóság az, hogy minél régebb óta fotózol Nikonnak, annál kevésbé szeretnéd az FTZ-t vagy az FTZ II-t, mert egyre kevésbé működnek az idősebb lencsékkel. (…) Nem tesz a Nikon nagy örökségének eleget.”\cite{Nikon_FTZ_Review}.