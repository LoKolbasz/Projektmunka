\subsection{Használatos technológiák és eszközök}
Az AF adapter felépítése több részre bontható. Ezeknek kell egymással együtt működve a Nikon F bajonettes objektíveknek natívhoz hasonló teljesítményt biztosítaniuk a Nikon Z bajonettes fényképezőgépvázakon.

\subsubsection{Mikrokontroller}
Ez az egység végzi az adatok feldolgozását, a jelek átalakírását. Ezt a szerepet egy Arduino Mega 2560 Rev3 egység fogja ellátni\cite{parmar2017design}. Ennek az eszköznek az előnyei:
\begin{itemize}
    \item Alacsony energiafogyasztás (7-12V, 1A)\cite{arduino_docs}
    \item Nagy mennyiségű analóg bemenet (16db tű)\cite{arduino_docs}
    \item Nagy mennyiségű digitális ki és bemenet (54db tű)\cite{arduino_docs}
    \item PWM tűk az analóg jelgeneráláshoz (15db)\cite{arduino_docs}
    \item 9V-os akkumulátorral képes üzemelni \cite{arduino_docs}
    \item Részletes dokumentáció, és sok segédanyag
\end{itemize}
Hátrányai:
\begin{itemize}
    \item Gyenge processzor (ATmega2560 16 MHz)\cite{arduino_docs}
    \item Kevés memória (8KB SRAM, 256KB FLASH, 4KB EEPROM)\cite{arduino_docs}
    \item Tisztán analóg kimenet hiánya.
\end{itemize}
Azoknál az átalakításoknál, ahol az interfésznek legalább az egyik felén digitális jel van, ott használatos ez az eszköz.
\paragraph{Digitális-Digitális transzformáció}
Ebben az esetben a be és kimeneteket a Digitális be, és kimenetekre kell kötni. A transzformációt a bemeneti jel fogadása után elvégzi az eszköz, és a kapott adatokat a kimeneti digitális kapcsolaton továbbküldi.
\paragraph{Analóg-Digitális transzformáció}
Ebben az esetben az egyik analóg bemenetre rákötjük az analóg interfészt, a digitális kimenethez pedig a kimeneti, digitális intefészt kötjük. Miután az 5.2.2-es szekcióban ismertetett módon létrehozzuk a digitális adatokat, azokat a kimeneti digitális tűn továbbküldjük.
\paragraph{Digitális-Analóg transzformáció}
Ilyenkor a digitális bemenet egy digitális bemeneti tűre van kötve, az analóg kimenet pedig egy PWM tűn fog megjelenni. Az 5.2.2-es szekcióban ismertetett eljárás alapján kiválasztjuk a függvényt, majd annak felhasználásával létrehozunk egy olyan modulált jelet a PWM csatornán, amely azt a függvényt reprezentálja. Ezt a jelet küldjük a kimeneti csatlakozásra.
\paragraph{Analóg-Analóg transzformáció}
A bemenet ilyenkor az egyik analóg bemeneti tűre van rákötve, a kimenet pedig egy PWM tűre. Az analóg bemenetből az 5.2.2-es szekcióban ismertetett módszerek segítségével kell előállítani a kimeneti függvényt, majd azt a \textbf{Digitális-Analóg transzformáció} bekezdésben ismertetett módon kell a PWM kimeneten megjelentetni. Ahol a kettő jel között matematikai, és nem logikai kapcsolat van, ott az alábbiakban ismertetett megoldás jobb eredményeket adhat.
\paragraph{Analog-Digital Converter - ADC (Analóg-Digitális Konverter)}
A mikrokontroller analóg lábain a beolvasásra kerülő jeleket a rajta található ATmega2560 processzor\cite{arduino_docs} Analóg-Digitális konvertere 10-bit es egész számokká alakítja\cite{ATmega_processor_datasheet}. Az így kapott szám azt mutatja, hogy a beolvasott érték az analóg lábak határértékei (0V és 5V) között hol helyezkedik el. Ebből feszültséget a kövekező formulával számolhatunk:
\begin{align}
    U(x) = \frac{5Vx}{1023}
\end{align}
\small (1023 a legnagyobb 10-bit-en reprezentálható érték)

\paragraph{Áramforrás}
Áramforrásként a  mikrokontroller dokumentációjában említett 9V-os akkumulátor használatos a prototípus fejlesztésekor. Amennyiben a váz képes az általa kibocsájtott, kiegészítőknek és objektíveknek szánt áramellátással, annak transzformálásával, vagy anélkül a mikrokontrollernek 7-12V-os\cite{arduino_aram} áramellátást biztosítani, akkor a végső implementáció használja azt. Ezzel el lehet kerülni az akkumulátor rendszeres cseréjét. A 7-12V-os áramot ilyenkor rá kell kötni a mikrokontroller VIN tűjére. Ennél ügyelni kell arra, hogy nem rendelkezik fordított polaritás védelemmel, ezért a bekötést megfelelően kell elvégezni.\cite{arduino_docs} Amennyiben a feszültség stabil 5V, a feszültség az elektromos áram az eszköz számára megfelel, így egyenesen az 5V-os tűkre köthető. Ekkor, a bemeneti áramtól függően, vagy rá kell kapcsolni az objektív áramellátását a váz által nyújtott forrásra, vagy a mikrokontroller PWM csatlakozásait használva kell azt árammal ellátni.

\paragraph{Programozása}
A mikrokontroller programozásához telepíteni kell az Arduino szoftvert a programozó számítógépére. A mikrokontrolleren nem lehet azt programozni. A programozáshoz elérhető egy telepíthető IDE és egy webes verzió is. A kettő közül a telepíthető verzió használata ajánlott. Ezt követően az arduinót a számítógéphez csatlakoztatjuk USB-n keresztül, így lehetőség nyílik a mikrokontrollerre programot feltölteni. \cite{monk2023programming} Ilyenkor az arduinokhoz készített programozási nyelvet használjuk. Ezzel a legegyszerűbb programozni ezeket az eszközöket, mivel az IDE a hardver köré épült (pl.: GUI vezényli az áttöltést, fordítást). A nyelv a C++ nyelvre alapszik, így a legtöbb C++-ban írt program az arduino fordítójával is a várt eredményt produkálja. A nyelv nem garantálja a memóriabiztonságot, aminek következtében a szoftver abnormálisan érhet véget. Ennek elkerülése érdekében kettő lépést is meg lehet tenni.
\begin{itemize}
    \item A dinamikus memóriaallokációk számának csökkentése a programban.
    \item Memóriabiztos rendszerprogramozási nyelv használata.
\end{itemize}
A második kritérium megvalósítására ideális programozási nyelv a Rust, amely egy memóriabiztonságos nyelv, azonban egy alapértelmezett projekt felépítése nem alkalmas az arduinón történő futtatásra. Ezért a projektet úgy kell konfigurálni, hogy a fordítás a mikrokontroller által végrehajtható bináris fájlt generáljon. A projektek menedzselésére, a legjobb eszköz a cargo, amely a Rust csomagkövetője\cite{cargo}. Ezáltal telepítjük a cargo-generate csomagot, amely képes már létező git tárolókat mintaként kezelve új Rust projektet létrehozni\cite{cargo-generate}. Ez által a csomag által hozzuk létre a projektet, a "github.com"-on megtalálható "avr-hal-tamplate"\cite{templates} tároló felhasználásával. A mikrokontrollerre az "avrdude" nevű programmal lehet telepíteni az így elkészült hexadecimális kódot, az elkészült fájl, a céleszköz, és az eszköz által használt ki, bemeneti csatlakozás megadásával.\cite{avrdude}

\subsubsection{Dedikált áramkörök}
Ezeket az \textbf{Analóg-Analóg} transzformáció során kell használni a gyorsabb jelátadás érdekében, ahol lehet. Itt a megfelelő matematikai műveleteket megvalósító áramkörök segítségével kell elvégezni a jelátalakítást. Ehhez szükséges a két jel közti matematikai összefüggés, összefüggések ismerete. Ahol ez a feltétel nem áll fenn, ott a fentebb leírt mikrokontrollerrel történő megoldást kell alkalmazni. A lehető legtöbb helyen passzív áramköröket kell alkalmazni, viszont ahol elkerülhetetlen, ott külső áramforrásra kell támaszkodni. Ekkor a prototípus fejlesztésénél a mikrokontrollertől független, a kész eszköznél lehetőleg a mikrokontroller által használt áramforrás használatos. Hasonló rendszerek használatával a fényképezőváz által nyújtott áramforrást el lehet osztani a mikrokontroller és az objektív között.

\subsubsection{AF Autofókuszmotor}
Ez által valósul meg az AF és AF-D objektíveken az Autófókusz. Ennek érdekében a mikrokontroller értelmezi a váz által küldött autófókuszinformációt (amely valószínüleg csak az irányt tartalmazza), és az alapján mozgat egy léptető motort. Amennyiben a csatlakoztatott objektívnek nem ad autofókusz utasításokat a váz, annyiban az adapternek olyan hamis információkat, azonosítási adatokat kell a váznak küldenie, amelyek hatására az megkezdi azok küldését. "Egy léptetőmotor egyszerre egy lépést tud mozdulni a hagyományos motorokkal ellentétben, amik folyamatosan mozognak."\cite{parmar2017design} Ezt a technológiát Nikon objektívek autofókuszálására is használják az AF-P jelölésű objektíveken \cite{Nikon_AF} a precizitásuk és sebességük miatt. "Ha parancsot adunk egy léptetőmotornak, hogy tegyen meg egy bizonyos számú lépést, akkor inkrementálisan annyi lépést mozog és megáll. "A léptető motor ezen alap természetének köszönhetően, általában alacsony költségű, nyílt hurkú pozíció irányitási rendszerekben használatosak. A nyílt hurkú irányítás azt jelenti, hogy nem igényel visszacsatolási információt a pozícióról." \cite{parmar2017design} Mivel nem tudjuk minden AF objektív jelenlegi fókusztávolságát\cite{Nikon_AF}, ezért egy ilyen nyílt hurokkal dolgozunk, tehát egy ilyen típusú motor ideális választás a fókuszcsavar pozícionálására.
% A stepper motor can moves one step at a time, unlike those conventional motors, which rotate continuously
% If we give command a stepper motor to move some exact number of steps, it rotates incrementally that many number of steps and stops. Because of this basic nature of a stepper motor, it is generally used in low cost, open loop position control systems
\paragraph{M542 Léptető Drive}\cite{parmar2017design}
A léptetőmotor irányítására használt egység. Képes kettő és négy fázisú léptetőmotorok irányítására, miközben azok felhevülése minimális marad. A mikrokontroller ennek adja át az utasításokat, melyek tartalmazzák, hogy melyik irányba, hány lépést kell megtennie a motornak.\cite{parmar2017design} A mikrokontroller PWM tűire kell csatlakoztatni a bemeneteit a drive használati utasítása szerint\cite{drive}.

\subsubsection{Váz oldali autofókuszcsavar}
Ennek az elemnek két egyszerű beszerzési módja van. A 3D-s nyomtatás, valamint egy ilyen rendszerrel rendelkező kameravázból való kiszerelés. Az utóbbi biztosabb működést ad, míg az előbbivel több példányhoz juthatunk hozzá, a megfelelő dizájn elkészítésének az árán. Ezt  a csavart egy rugó nyomja ki az adapterből, annak érdekében, hogy az objektívek, amelyek ezt nem használják is csatlakoztathatóak legyenek. Emiatt a szárán lévő fogaskeréknek olyannak kell lennie, hogy amikor mozog, akkor is csatlakoztatva maradhasson a léptetőmotorral, amivel fogaskerekek kötik össze.

\subsubsection{Adaptertest}
Az adapter két bajonettzárat összekötő testének a megfelelő távolságra kell eltartania az F bajonettes objektívet a Z bajonettes váztól, annak érdekében hogy az objektív a specifikációinak megfelelő intervallumon tudjon fókuszálni. Ennek érdekében egy már létező "buta" adaptert módosítunk a fentiekben leírt elektronikus komponensekkel. Ez az UPC:847372046812 termékkódú Fotodiox Pro Adapter Nikon F záras G-típusú objektíveket Nikon Z záras tükörnélküli kamerákra adaptáló adapter. Ennek előnye, hogy a G típusú objektívekről lehagyott rekeszátmérő gyűrűt tartalmazza, ezért akár annak mozgatásával is állíthatja a mikrokontroller a rekeszátmérőt. Elektronikus kapcsolatokat nem tartalmaz, ezért azokat kézzel kell elhelyeznünk rajta. Ehhez 3D nyomtatott elemeket rögzítünk az adapter csatlakozásának belsejébe. Ezeken fúrt lyukak találhatóak, amin keresztül az elektromos jeleket vezető drótok csatlakoznak a vázhoz, vagy az objektívhez. Mivel nem tartalmaz autofókuszcsavart (\ref{fig:F100_AF_csavar}), ezért az F bajonettes csatlakozó karimájába egy lyukat kell fúrni annak. Mivel nincs sok hely az adapteren belül, ezért az elektronikus komponenseket a Fotodiox adapter külsejére rögzítjük, és a jelszállító drótokat az adapterbe fúrt lyukon vagy lyukakon keresztül vezetjük ki úgy, hogy a rést ezt követően fénymentesen lezárjuk. Ezen kívül az adaptertest oldalába még egy lyukat fúrunk, amin keresztül a csavar alapú atófókuszrendszer csavarja a meghajtást fogja kapni.
