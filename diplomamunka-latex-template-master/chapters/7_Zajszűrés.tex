Annak érdekében, hogy egyszerűbb legyen egyezéseket találni az analóg bemenetek, és az azokhoz eltárolt minták között.
Ezek a szűrők az elemzési fázisban is jól használhatóak, így könnyebb automatizált eszközökkel mintákat találni a jelekben.

\subsection{Digitális zaj csökkentése}
"Tipikus forrásai a magas frekvenciájú zajnak, amely hatással van az áramkörökre és elektronikus rendszerekre: digitális logika, kapcsoló tápegységek, elektrosztatikus kisülések, motorok és relék, [...]. 
A digitális zaj valósszínűleg a legelterjettebb zajforrás az elektronikus rendszerekben."\cite{smith1992high}
Az Arduino Mega 2560 az ADC pontosságának növeléséhez nyújt egy "ADC Noise Reduction" alvási módot \cite{arduino_at_mega_datasheet}. 
"Az ADC képes Alvó módban üzemelni, de ehhez az ADC-nek a dedikált ADCRC-r kell használnija órajel forrásként. 
Amikor az ADCRC van kiválasztva órajel forrásként, az ADC hardver vár még egy instrukciós ciklust ($T_{CY}$) mielött megkezdi a konverziót. 
Ez megengedi, hogy a SLEEP instrukció végre legyen hajtva, amely csökkentheti a rendszerzajt a konverziós folyamat során."\cite{ATmega_processor_datasheet}

\subsection{Zajcsökkentés DCT és DSC szótárakon alapú OMP algoritmussal}
