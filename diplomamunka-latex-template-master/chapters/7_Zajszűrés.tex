"A zaj egy kéretlen jelek egy típusa.[...] Az adatgyűjtési folyamat során a zaj a jelekben perzisztens.[...] 
Még az analóg-digitális átalakítás is zajt ad az átalakított jelhez. 
A konverter minőségétől és a jel felhasználásától függően ennek jalantős hatással lehet."\cite{noise_reduction_omp}
Annak érdekében, hogy egyszerűbb legyen egyezéseket találni az analóg bemenetek, és az azokhoz eltárolt minták között.
Ezek a szűrők az elemzési fázisban is jól használhatóak, így könnyebb automatizált eszközökkel mintákat találni a jelekben.

\subsection{Digitális zaj csökkentése} \label{ADC_noise}
"Tipikus forrásai a magas frekvenciájú zajnak, amely hatással van az áramkörökre és elektronikus rendszerekre: digitális logika, kapcsoló tápegységek, elektrosztatikus kisülések, motorok és relék, [...]. 
A digitális zaj valósszínűleg a legelterjettebb zajforrás az elektronikus rendszerekben."\cite{smith1992high}
Az Arduino Mega 2560 az ADC pontosságának növeléséhez nyújt egy "ADC Noise Reduction" alvási módot \cite{arduino_at_mega_datasheet}. 
"Az ADC képes Alvó módban üzemelni, de ehhez az ADC-nek a dedikált ADCRC-r kell használnija órajel forrásként. 
Amikor az ADCRC van kiválasztva órajel forrásként, az ADC hardver vár még egy instrukciós ciklust ($T_{CY}$) mielött megkezdi a konverziót. 
Ez megengedi, hogy a SLEEP instrukció végre legyen hajtva, amely csökkentheti a rendszerzajt a konverziós folyamat során."\cite{ATmega_processor_datasheet}

\subsection{Zajcsökkentés DCT és DSC szótárakon alapú OMP algoritmussal} \label{OMP}

Ez a módszer a fehér nagyjából 34dB-el képes csökkenteni, miközben visszaállítja torzításmentesen az eredeti jelet.\cite{noise_reduction_omp}
Az eljárás "a feldolgozott jelből diszkrét koszinus transzformációval és diszkrét szinusz transzformációval létrehozza annak a ritka reprezentációját. Az ortogonális egyezésre törekvés algoritmusa rekonstrukciós algoritmusként van használva, hogy a zaj nélküli jel bázisait megtaláljuk [...]."\cite{noise_reduction_omp}
A ritka reprezentáció a következő lineáris kombinációval képes a jelet reprezentálni:
\[x[n]=\sum_{m=1}^{M}D[n,m]\alpha[m]+r[n]\]
ahol $x[n]$ egy $x \in \mathbb{R}^{N \times 1}$ jel n-edik mintája, $D[n,m]$ az n-edik mintája az m-edik oszlopnak (atom) egy $D \in \mathbb{R}^{N \times M}$, $a[m]$ az m-edik együtthatója egy $a \in \mathbb{R}^{M \times 1}$ritka vektornak (van nem 0 eleme) és $r[n]$ az n-edik mintája a visszamaradó $r \in \mathbb{R}^{N \times 1}$ reprezentációnak.\cite{noise_reduction_omp}

\paragraph*{OMP(ortogonális egyezésre törekvés) algoritmus}
"Egy mohó algoritmus, amely használható jelvisszaállításra hiányis együtthatókból tömörített érzékelésnél."\cite{noise_reduction_omp} Esetünkben a jel dekompozíciójára használjuk.

% \begin{algorithm}
%     \caption{OMP}\label{alg:cap}
%     \begin{algorithmic}
%         \Require
%         \Ensure
%         \While {$\neg STOP\_CRITERION$}
%             \State $j++$
%             \State $i = \underset{m=\{1,...,M\}}{argmax}| \langle r, D \rangle |$ \Comment{A maximum abszolút értékű belső szorzata a \newline visszamaradó jelnek ($r$) és a $D$ szótár összes $d_m$ atomjának.} % https://mathworld.wolfram.com/AngleBracket.html
%             \State $U_{j}=[U_{j - 1} \cup i]$
%             \State $D_{j}=[D_{j - 1} | d_{i}]$
%            \State $\underset{\alpha}{min} \norm{x - D_{j}\alpha_{j}}_{2}^{2}$
%           \State $r_{j} = x - \tilde{D}_{j}\alpha{j}$ \Comment{$\alpha = \tilde{D}_j^{\dagger}x$ ahol $\dagger$ Moore–Penrose pseudo-inverz mátrixot jelez}
%        \EndWhile
%        %\Return \alpha_{j}, U_{j}, r_{j}
%    \end{algorithmic}
%\end{algorithm}\cite{noise_reduction_omp}
% Ahol: 
% $x$: dekompozálandó jel
% $D$: szótár, amely regisztrálja a kiválasztott atomot
% $STOP_CONDITION$: Leállási feltétel
% $r$: Visszamaradó jel.(Inicializálásnál ($r_0$) az eredeti jel, ekkor $D_0$ és $U_0$ is üres)
% $U$: index vektor, amely a kiválasztott atom indexét regisztrálja.
% $j$: Elleniterátor. (Kiinduláskor 0)
% $\tilde{D}$: Atomok részmátrixa.
A ritka reprezentáció megtartása érdekében $M > N$ nek igaznak kell lennie és a szótárban legalább $N + 1$ atomank kell lennie.
A diszkrét fourier függvény complex része elkerülése érdekében a DCT és a DST algoritmusokat használjuk. Ezek eredményéiből jön létre a szótár: \[D = [C|S]\]
A koszinuszok a DCT-II transformáció alapján: \[C[n, k] = \beta_c \cos[\frac{\pi k (2n + 1)}{2N}]\]
A szinuszok a DST-II transformáció alapján: \[C[n, k] = \beta_s \sin[\frac{\pi (k + 1) (2n + 1)}{2N}]\]
Ahol 
\begin{equation}
    \beta_c =
    \begin{cases}
        \sqrt{\frac{1}{N}}, n = 0\\
        \sqrt{\frac{2}{N}}, 1 \leq n \leq N - 1
    \end{cases}
\end{equation}
\begin{equation}
    \beta_s = 
    \begin{cases}
        \sqrt{\frac{2}{N}}, 0 \leq n \leq N - 2\\
        \sqrt{\frac{1}{N}}, n = N - 1
    \end{cases}
\end{equation}
$n = {0, 1,..., N - 1}$, $k = {0, 1,..., N - 1}$ és $N$ a jelminták száma. Ezek alapján a ritka reprezentáció:
\begin{equation}
    x[n] = \sum_{p=1}^N C[n,p]\alpha_c[p] + \sum_{q=1}^N S[n,q]\alpha_s[q] + r[n]
\end{equation}
Ahol $\alpha_c \in \mathbb{R}^{N \times 1}$ a koszinuszok, $\alpha_s \mathbb{R}^{N \times 1}$ pedig a szinuszok együttahtói amik a $C \in \mathbb{R}^{N \times N}$ és a $S \in \mathbb{R}^{N \times N}$ mátrixokhoz tartoznak. A szótárban $2N$ atom lesz.\cite{noise_reduction_omp}
\\
\begin{algorithm}
    \caption{OMP zajcsökkentés}\label{alg:cap}
    \begin{algorithmic}
        \Require $C, S, x, STOP\_CRITERION$
        \Ensure $r_0 = x, D_0 = {}, j = 0$
        \While {$\neg STOP\_CRITERION$}
            \State $j++$
            \State $i = \underset{m=\{1,...,M\}}{argmax}(\sqrt{\langle r_{j-1}, C_m \rangle^2 + \langle r_{j-1}, S_m \rangle^2})$ \Comment{$\langle \bullet , \bullet \rangle$ skalárszorzat} % https://mathworld.wolfram.com/AngleBracket.html
            \State $U_{j}=[U_{j - 1} \cup {i, N + i}]$
            \State $D_{j}=[D_{j - 1} | {C_{i}, S_i}]$
            \State $\alpha = \tilde{D}_j^{\dagger}x$ ahol $\dagger$
           \State $\underset{\alpha}{min} \norm{x - D_{j}\alpha_{j}}_{2}^{2}$
          \State $r_{j} = x - D_{j}\alpha{j}$
       \EndWhile
       \State \Return $\alpha_{j}, U_{j}, r_{j}$
   \end{algorithmic} \cite{noise_reduction_omp} 
\end{algorithm}
\\
Ahol: \\
$x$: dekompozálandó jel. \\
$S$: Szinusz komponensek. \\
$C$: Coszinusz komponensek. \\
$D$: Szótár, amely a $[C|S]$ mátrixból áll. C és S elemeknek párban kell lenniük (ugyan akkora az elemszámuk).\\
$STOP_CONDITION$: Leállási feltétel. \\
$r_j$: Visszamaradó jel a j-edik iterációnál.(Inicializálásnál ($r_0$) az eredeti jel, ekkor $D_0$ és $U_0$ is üres). \\
$U_j$: A j-edik iterációnál kiválasztott atom indexek halmaza. \\
$j$: Elleniterátor. (Kiinduláskor 0) \\
$\tilde{D}$: Atomok részmátrixa. \\

\paragraph{Megállítási feltétel}
A megállítási feltételhez, egy a szimulált lehűtésnél is alkalmazható módszert használunk.
Ennek alapján akkor állítjuk le az iterációt, ha a az algoritmus energiaszintje elért egy minimumot. Ezt az OMP esetében az alábbiak szerint becsüljük meg:
\begin{equation}
    E_m = \sqrt{\frac{\alpha_m^2 + \alpha_{N + m}^2}{2N}}
\end{equation}
Az $E_min$ értéke fehér zaj esetén meghatározható annak a konstans teljesítményspektrum sűrűsége miatt. Viszont a módszer miatt az energiaszint elérheti a minimumot, de a megoldás még hullámzik. Ennek elkerülése érdekében, csak akkor érjen véget a ciklus, ha legalább már kettő egymás utáni megoldás energiaszintje a határérték alatt van.\cite{noise_reduction_omp}