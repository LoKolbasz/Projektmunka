\label{ora}

A Hardver Absztakciós Rétegbe (HAL) nincs beépítve \[reference here\], ezért a pontos időzítések számításához egy óra manuális implementálása szükséges.
Ehhez egy számláló regiszter használatos. A számláló regiszter minden óraciklusban növeli a benne tárolt értéket, így az óra frekvenciájának ismeretében a számláló indítása óta eltelt idő kiszámítható a következő formulával: \[f_{clock} \cdot c\]
Behelyettesítve:\[\frac{c}{t}\]
Ahol:
\captionsetup[table]{list=no}
\begin{table}[H]
    \begin{tabular}{ll}
    $t$:    &periódusidő \\
    $c$:    &számláló
    \end{tabular}
\end{table}	
\captionsetup[table]{list=yes}

\subsection{Újrahasználhatóság}

Annak érdekében, hogy az órát később is fel lehessen használni (függőségi injekcióval), célszerű egy interfész (trait) implementálása. Egy ilyet szolgáltat az "embedded time" csomag, ami a korábban említett előny mellett különböző funkcionalitásokkal is bővíti a programot, mint például mértékegység átváltások, időzítők, időhöz tartozó adattípusok.

\subsection{Számláló regiszter}

Az eszköz több számláló regisztert is tartalmaz. Ezek között a legjelentősebb különbség, a számláló bitmélysége.
A szakdolgozat során a 8 bites TC0 regiszter volt használva, annak érdekében, hogy a 16 bites regiszter szabad maradjon egy esetleges fontosabb funkció betöltésére.

\subsection{Túlcsordulás}

Annak érdekében, hogy használható hosszúságú időintervallumok tárolhatóvá váljanak, szükséges egy túlcsordulásszámláló használata.
Ennek bitmélysége befolyásolja azt, hogy mikori lesz az indítás után legkésőbbi pontosan olvasható időpont.
A jelenlegi idő a túlcsordulás számlálóval a következő:
\begin{equation}
    \frac{c_{reg} + c_{of} \cdot c_{reg\_max}}{t}
    \label{eq:ido_szamlalokbol}
\end{equation}
Ahol:
\captionsetup[table]{list=no}
\begin{table}[H]
    \begin{tabular}{ll}
    $c_{reg}$:&A számláló regiszterben tárolt érték \\
    $c_{of}$:&Túlcsordulás számláló \\
    $c_{reg\_max}$:&A számláló regiszter legnagyobb lehetséges értéke \\
    \end{tabular}
\end{table}
\captionsetup[table]{list=yes}
A túlcsordulásszámláló inkrementálása a processzor megszakításainak használatával történik.
A használt megszakítás az óra paraméterei alapján kerül kiválasztásra.

\subsection{Számláló regiszter konfiguráció}

\subsubsection*{Skálázó} Órajelenként mennyivel növelje a számlálót.
\begin{table}[H]
    \centering
    \begin{tabular}{|lr|}
    \hline
    \rowcolor{lightgray}
    \multicolumn{2}{|c|}{Lehetséges értékek}                              \\ \hline
    \multicolumn{1}{|l|}{0}    & direkt / nincs skálázás                  \\ \hline
    \multicolumn{1}{|l|}{8}    & \multirow{4}{*}{skálázás adott értékkel} \\ \cline{1-1}
    \multicolumn{1}{|l|}{64}   &                                          \\ \cline{1-1}
    \multicolumn{1}{|l|}{256}  &                                          \\ \cline{1-1}
    \multicolumn{1}{|l|}{1024} &                                          \\ \hline
    \end{tabular}
    \caption{Skálázó lehetséges értékei}
    \label{tab:skalazo_ertekek}
\end{table}

\subsubsection*{Túlcsordulási érték} A számláló maximális értéke. Ha a számláló eléri ezt a számot, az minősül túlcsordulásnak.
Lehetséges értékek:
\captionsetup[table]{list=no}
\begin{table}[H]
\begin{tabularx}{\textwidth}{@{}lL@{}}
    255       & Normál mód: Regiszter túlcsordulás megszakítás. A regiszter túlcsordulását használja fel a megszakítás hívásához.     \\
    < 255     & CTC mód: Komparátor megszakítás. Amikor a regiszter értéke eléri ezt a számot, akkor hívódik meg a megszakítás.          \\
    \end{tabularx}
\end{table}
\captionsetup[table]{list=yes}
Mivel az óra frekvenciája konstans, ezért a megszaskítások gyakorisága ennek a kettő változónak használatával állítható az óra példányosításánál.